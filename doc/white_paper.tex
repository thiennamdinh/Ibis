\documentclass{article} \usepackage{parskip}
\usepackage{titlesec}
\usepackage{times}

\title{\vspace{-5ex}Ibis}
\date{\vspace{-10ex}}

\titleformat{\section}[block]{\bfseries\filcenter}{}{1em}{}

\begin{document}
\maketitle
\noindent\rule{\textwidth}{1pt}

Imagine a day when the so-called blockchain revolution has long subsided, when
the waves of technological innovations born today had sparked, matured, and
faded into the domain of routine network maintenance. In this world, Alice owns
a token for just about everything. From power and transportation to web browsing
and prediction markets, all of these tokens will serve to grease the wheels of a
much more efficient and profitable global economy.

Profit, however, is not the only story of this future. On certain days, when
Alice feels the desire to contribute to her favorite philanthropic cause, this
will be done with a dedicated token. When she gives a small gift to her friend
so he can in turn donate it to his favorite charity, this too will take place on
the same medium. Sooner or later, for better or for worse, the age of digital
money will operate in no small part due to the flow of a single ubiquitous
charity currency. If this currency is to be synonymous with altruism, if it can
truly tokenize the notion of human decency, it must be constructed in not only
the right way, but for the right reasons. In this way, for these reasons, we
humbly present our proposal for Ibis.

\section{The Ibis Token}

The Ibis token is an Ethereum-based cryptocurrency used for charitable donations
on the blockchain. The core concept is exceedingly simple: it is a token which
is pegged 1:1 with ether that is guaranteed to be eventually given to charity. A
set of smart contracts maintains an ERC20 token that obeys the following rules:

\begin{itemize}
\item Ether may be exchanged for Ibis tokens by any user
\item Ibis tokens may be sent to any address
\item Ibis tokens may only be exchanged for Ether by a charity address
\end{itemize}

The Ibis organization is responsible for managing the contracts and for
maintaining the public white-list of accepted charities. Initially, the
white-list will be composed only of well-recognized names within the nonprofit
and cryptocurrency communities. It will be expanded to include smaller and more
niche organizations as Ibis gains credibility. In the initial stages, we plan
to release a custom user application and will commit heavily to outreach and
market research. However, this initial progress will not be sustainable. Ibis
does not launch with an initial coin offering. It does not charge taxes or
fees. We are a nonprofit in the strictest sense of the word, relying exclusively
<on volunteer hours and donations. As such, Ibis aims to be the most minimalistic
platform possible that can still embody the concept of a charity currency.

The full potential of Ibis will only be realized the help of a strong community
of developers, altruistic entrepreneurs, and everyday users who share our common
ideals. At the end of the day, we are not in the business of selling
technological innovations or financial prowess. We are in the business of
selling a simple story, a story that comes in three acts.

\section{Stage 1: Tokenized Altruism}

In its initial phase, Ibis will consolidate the types of charitable transactions
that are already present in the modern economy. The benefits of moving
philanthropy to the blockchain will be much the same as the benefits in any
other sector: security, transparency, and standardization.

Corruption, mismanagement, and outright theft are as much a problem in the
nonprofit sector as anywhere else. By moving charitable donations to the
blockchain, we can gain a better insight into the flow of money and be alerted
to many forms of abuse. A public record of donations also provides an immensely
useful body of documentation for important purposes such tax deductions. Once
the Ibis ledger becomes a legally recognizable standard, it will be an
attractive platform for everything from corporate giving programs to disaster
relief efforts.

From the perspective of a common user, Ibis is especially useful in situations
where money needs to change hands before it is donated to charity. For instance,
donations are often made in lieu of flowers at a funeral. A charitable school
fund-raiser must deduct costs before sending out the proceeds. As these use
cases become more and more common, the end users themselves will help to set the
stage for the next development in the Ibis ecosystem.

\section{Stage 2: A Shadow Economy}

Whereas as the first stage represents improvements to the status quo, the second
stage promises something decidedly new. With the help of a new form of money
that is synonymous with the concept of altruism, we can address a set of
problems that, until now, have been largely neglected. Consider the following
situations:

\begin{itemize}
\item A man would like to pay his close neighbor to help him move but decides
  that this exchange of money among friends would go against the social norm.
\item A woman, faced with the need to dispose of her deceased love one's
  possessions, is hesitant to sell them off for money, which she feels may be too
  superficial.
\item An entrepreneur creates a useful service allowing drivers who are leaving
  crowded parking spaces to sell their vacant spot to newcomers. He is run out of
  town for having the audacity to monetize public parking resources.
\end{itemize}

The story is the same in all of these situations: market failure. There is a
very real and prevalent class of market failures that stems from the disconnect
between the perfectly rational predictions of economic models and the complex
social and moral norms which govern society. These markets fail in part because
of a view that money is based, dirty, and symbolic of the worst aspects of human
greed.

Ibis can change all of this. Suppose that instead of traditional money, Ibis
tokens were used to facilitate each of the transactions in our example
situations. All that is needed is for parties on one side of the transaction
(the neighbor, the woman, the vacating driver) to see some amount of value in
generating money for their favorite charity. On the opposite side of the
transaction, any normal self-interested individual in need of the good or
service would be happy to make the payment in whatever medium is requested.

Ibis represents a method of exchange that, in certain situations, is as fungible
and valuable as any other. However, by turning on its head the negative
connotations of money, Ibis can circumvent a tangle of social and moral
frictions to facilitate a new class of economic activity that otherwise would
have never occurred.

Stage two imagines a world in which these scenarios are not stories in a
white-paper but mundane features of everyday life. It imagines a world where
Ibis casinos, Ibis ticket scalpers, and any number of other novel
donation-generating industries are now possible. With the rise of these new
business endeavors, not only would more money flow to philanthropy, but actual
GDP measures would rise with a new wave of previously unattainable products and
services. There is an entire shadow economy out there that has never before been
monetized, one that Ibis has the ability to break loose.

\section{Stage 3: Goodwill Capitalism}

The final nature of a fully fledged universal charity currency will not be clear
for some time. However, as difficult as it is to predict this future, it is just
as exciting to imagine.

One distinctly intriguing possibility is the notion of charity
corporations. Charities today are naturally classified as nonprofits. As such,
they do not have access to the venture capital or investment opportunities
available to their for-profit counterparts. That may change in a post-Ibis
world. Suppose that charity startups can now sell equity shares in their
organization. Philanthropic investors would be able to buy this equity using
Ibis tokens. Any future dividends from excess donations and capital gains
obtained on secondary markets would again be paid out in Ibis. The result is a
fully fledged financial ecosystem in which every piece of money enters in one
direction and, one way or the another, eventually ends up in the service of
philanthropy. In between, it is free to facilitate the beneficial processes of
modern capitalism.

If the Ibis financial sector grows large enough, we might see Ibis loans, Ibis
interest rates, and even Ibis ICOs for new and novel charity tokens. There might
be mutual funds and indexes for investing in various classes of
charities. Perhaps in this world, we will witness a new class of serial Ibis
entrepreneurs and charity moguls each competing for fame and social influence,
this time in a race to the top for the benefit of society as a whole.

We certainly wouldn't go so far as to suggest that any of these developments are
inevitable or even likely. But if such far-fetched ideas are even plausible
today, what else might a decade of real progress and societal ingenuity bring?

\section{Ethereum}

The path to a better future begins with the design decisions that are made
today. We have decided to launch on Ethereum because there is no better setting,
now or at any point in history, that is more conducive to the success of our
proposal. The standard practices of the cryptocurrency space provides an
immediate way for users to manage their own money while alleviating a large
amount of technological and legal responsibilities on the part of the Ibis
organization. Furthermore, Ethereum offers the ability to exploit truly
programmatic money. From networks of trust and reputation schemes for charities
to variably valued donation tokens, we can to build an entirely new class of
tools that will allow us to consistently push toward better security,
efficiency, and utilization.

We will also have access to the full arsenal of general protocols and
applications available to all ERC20 tokens. Ibis will succeed not because it is
somehow the best token, but because it is only one of many in this space, one
that by all accounts needs to exist. This argument, we feel, is more compelling
than any rhetoric of disruption and revolution.

\section{Conclusion}

One way or another, society will someday operate with the help of an ubiquitous
dedicated charity currency. It is imperative that this resource will not be
owned by the for-profit company or special interest group with the deepest
pockets, but by the community with the most genuine intentions.

We claimed that Ibis was conceived in the right way and for the right
reasons. The right way to build it, then, is through a community effort. For us,
this is the only way. If you have experience to offer, we would gladly welcome
any suggestions you may have, code contributions, and help in spreading
awareness. If believe you can improve upon the concept in any way, then we
sincerely hope that you pursue your endeavor to its full potential and consider
reconvening with us when the time comes. There are no competitors in this
space, only progress.

As Ibis moves forward, step in step with the greater waves of change around it,
its worth will be measured by a dedication to the ideals and vision presented
here. The hope is that this document has served as some small, muddled window
into a worthwhile future. But we are no longer building a window, we are
building a door. What stories lie on the other side, we shall soon find out.

\end{document}
